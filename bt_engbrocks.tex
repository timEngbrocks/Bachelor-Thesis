%==============================================================================
% Fix Cite
% Find 1: (^(?!%).+)(\\cite\{.*\})
% Replace 1: $1\\iffalse $2 \\fi 
% Find 2: \\iffalse(.*)\s\\fi\s
% Replace 2: $1
% Fix Ref
% Find 1: (^(?!%).+)(\\ref\{.*\})
% Replace 1: $1\\iffalse $2 \\fi 
% Find 2: \\iffalse(.*)\s\\fi\s
% Replace 2: $1
%==============================================================================

%==============================================================================
% Moved here from header to be compatible with the VSCode Extension: Latex Workshop
\documentclass[
	pdftex,				% PDFTex verwenden da ausschliesslich ein PDF erzeugt wird.
	a4paper, %twoside,	% Verwenden von DIN A4 Papier.
	11pt,				% Grosse Schrift, besser geeignet für A4.
	parskip=half,		% Halbe Zeile Abstand zwischen Absätzen.
	numbers=noenddot,	% Keine Punkte hinter Nummern
	pagesize,           % Schreibt die Papiergroesse in die Datei 
	BCOR=10mm,			% Bindekorrektur
	DIV=13,				% Alternativ 12 oder 14
	headinclude,		% Kopfzeile in den Textbereich
	headsepline,		% Linie nach Kopfzeile.
	titlepage,
	headings=small,
	bibliography=totocnumbered,	% Bibliographie im TOC nummeriert
]{scrbook}
%==============================================================================

%==============================================================================
% Dokument einrichten und Pakete laden
%==============================================================================
\usepackage{pifont}
\newcommand{\cmark}{\ding{51}}
\newcommand{\xmark}{\ding{55}}

\usepackage{setspace}\setstretch{1.2} 
%\usepackage{titlesec}
\usepackage{longtable}

%\titlespacing\section{0pt}{12pt plus 4pt minus 2pt}{0pt plus 2pt minus 2pt}
%\titlespacing\subsection{0pt}{12pt plus 4pt minus 2pt}{0pt plus 2pt minus 2pt}
%\titlespacing\subsubsection{0pt}{12pt plus 4pt minus 2pt}{0pt plus 2pt minus 2pt}

\usepackage{scrhack}
\usepackage{tabularx}

\usepackage[printonlyused]{acronym}
\usepackage{multicol}

\usepackage[table]{xcolor}

\usepackage{comment}

%
% Zeichenkodieruung und Sprache
%

\usepackage[utf8]{inputenc} % Dokument
\usepackage[T1]{fontenc} % Schrift
\usepackage[ngerman, english]{babel}

\usepackage{eurosym}

%
% PDF-Metadaten setzen
%

\usepackage{pdfpages}
\usepackage[
	% Titel des PDF Dokuments
	pdftitle={BT Engbrocks: Development of a Concept for the Monitoring of Decentralized Identity Solutions Based on DevOps Concepts},
	% Autor des PDF Dokuments
	pdfauthor={Tim Engbrocks},
	% Thema des PDF Dokuments
	pdfsubject={Bachelor thesis},
	% Erzeuger des PDF Dokuments
	pdfcreator={Tim Engbrocks},
	% Schlüsselwörter für das PDF
	pdfkeywords={},
	% Dokumenttitel statt Dateiname anzeigen
	pdfdisplaydoctitle=true,																% Sprache des Dokuments
	pdflang=en,
	bookmarksopen=true,
	bookmarksdepth=1,
	colorlinks,
	linkcolor = black,
	citecolor=black,
	urlcolor=black,
]{hyperref}

%
%  Zusätzliche Pakete laden
%

% Anführungszeichen
\usepackage[style=american]{csquotes}

% erweiterte Tabelleneigenschaftn
\usepackage{array, ragged2e}

% Einbinden von Grafiken
\usepackage{graphicx}	

% mathematischer Textsatz
%\usepackage{amsmath}
%\usepackage{amssymb}
%\usepackage{dsfont}

% Textteile drehen
%\usepackage{rotating}	

% Farbpakete
%\usepackage{color}

% Quellcode sauber formatieren
\usepackage{listings}	

% Font 'Latin Modern Family' verwenden
\usepackage{microtype}
\usepackage{helvet}
\usepackage{mathptmx}

%==============================================================================
% Einstellungen und Definitionen
%==============================================================================

% Farben definieren

\definecolor{light-gray}{gray}{0.95}
%\definecolor{LinkColor}{rgb}{0,0,0.5}
%\definecolor{ListingBackground}{rgb}{0.85,0.85,0.85}
%\definecolor{CommentColor}{rgb}{0, 0.5, 0}
%\definecolor{StringColor}{rgb}{0.63, 0.09, 0.09}

% KOMA-Script Option, Zeilenumbruch bei Bildbeschreibungen.
\setcapindent{1em}

% Stil der Kopf- und Fusszeilen.
\usepackage[headsepline,automark,pagestyleset=KOMA-Script, markcase=ignoreuppercase]{scrlayer-scrpage}
\pagestyle{scrheadings}

% Stil der Überschriften auf normale Schrift.

%\setkomafont{sectioning}{\normalfont\bfseries}		 % Titel mit Normalschrift
\setkomafont{captionlabel}{\normalfont\bfseries}	 % Fette Beschriftungen 
%\setkomafont{pageheadfoot}{\normalfont\itshape}     % Kursive Seitentitel
\setkomafont{descriptionlabel}{\normalfont\bfseries} % Fette Beschreibungstitel

% Codelisting korrekt bezeichnet ausgeben
%\renewcommand\lstlistingname{Source code}

%==============================================================================
% Listings
%==============================================================================

\lstloadlanguages{% Check Dokumentation for further languages ...
  XML,
  HTML,
  Java,
  Tex
}


\lstset{
  %basicstyle=\scriptsize\ttfamily, % Standardschrift
	basicstyle=\footnotesize\ttfamily,
  %numbers=left, % Ort der Zeilennummern
  %numberstyle=\tiny, % Stil der Zeilennummern
  %stepnumber=2, % Abstand zwischen den Zeilennummern
	%numberblanklines=false,
  numbersep=5pt, % Abstand der Nummern zum Text
  tabsize=2, % Groesse von Tabs
  extendedchars=true, %
  breaklines=true, % Zeilen werden Umgebrochen
  %keywordstyle=\color{red},
  frame=b,
  % keywordstyle=[1]\textbf, % Stil der Keywords
  % keywordstyle=[2]\textbf, %
  % keywordstyle=[3]\textbf, %
  % keywordstyle=[4]\textbf, \sqrt{\sqrt{}} %
  %stringstyle=\color{white}\ttfamily, % Farbe der String
  showspaces=false, % Leerzeichen anzeigen ?
  showtabs=false, % Tabs anzeigen ?
  xleftmargin=17pt,
	xrightmargin=17pt,
  framexleftmargin=17pt,
  framexrightmargin=17pt,
  framexbottommargin=4pt,
  backgroundcolor=\color{white},
  showstringspaces=false % Leerzeichen in Strings anzeigen ?
}

\lstset{literate=%
    {Ö}{{\"O}}1
    {Ä}{{\"A}}1
    {Ü}{{\"U}}1
    {ß}{{\ss}}1
    {ü}{{\"u}}1 
    {ä}{{\"a}}1
    {ö}{{\"o}}1
    {~}{{\textasciitilde}}1
}

\usepackage{caption}
\DeclareCaptionFont{white}{\color{white}}
\DeclareCaptionFormat{listing}{\colorbox[cmyk] {0.43, 0.35, 0.35,0.01}{\parbox{\textwidth-2\fboxsep-2\fboxrule-0pt} {\hspace{15pt}#1#2#3}}}
\captionsetup[lstlisting]{format=plain, labelfont=bf}
%\captionsetup[lstlisting]{format=listing,labelfont=white, textfont=white,singlelinecheck=false, margin=0pt, font={bf,footnotesize}}

%
% code listing style
%

\lstdefinestyle{kit-cm}{
  backgroundcolor=\color{light-gray},
  belowcaptionskip=1\baselineskip,
  breaklines=true,
  frame=single,
  framexleftmargin=15pt,
  language=C,
  showstringspaces=false,
  basicstyle=\footnotesize\ttfamily, 
  numbers=left,                    
  numbersep=7pt,                
  numberstyle=\tiny\color{black},
  captionpos=b,
  keywordstyle=\color{blue}
}

\lstdefinelanguage{Swift}{
  keywords={associatedtype, class, deinit, enum, extension, func, import, init, inout, internal, let, operator, private, protocol, public, static, struct, subscript, typealias, var, break, case, continue, default, defer, do, else, fallthrough, for, guard, if, in, repeat, return, switch, where, while, as, catch, dynamicType, false, is, nil, rethrows, super, self, Self, throw, throws, true, try, associativity, convenience, dynamic, didSet, final, get, infix, indirect, lazy, left, mutating, none, nonmutating, optional, override, postfix, precedence, prefix, Protocol, required, right, set, Type, unowned, weak, willSet},
  ndkeywords={class, export, boolean, throw, implements, import, this},
  sensitive=false,
  comment=[l]{//},
  morecomment=[s]{/*}{*/},
  morestring=[b]',
  morestring=[b]"
}

\lstdefinelanguage{Gherkin}{
	morekeywords = {
		Given,
		When,
		Then,
		And,
		Scenario,
		Feature,
		But,
		Background,
		Scenario Outline,
		Examples
	},
	sensitive=true,
	morecomment=[l]{\#},
	morestring=[b]",
	morestring=[b]'
}


\lstdefinelanguage{json}{
     string=[s]{"}{"},
    stringstyle=\color{blue},
    comment=[l]{:},
    commentstyle=\color{black},
}

\DeclareCaptionFont{black}{\color{black}} 
\DeclareCaptionFormat{listing}
  {\colorbox{white}
     {\parbox{\dimexpr\textwidth-2\fboxsep}{\centering #1#2#3}}}
% \captionsetup[lstlisting]{format=listing,labelfont=black,textfont=black}

\usepackage{hyperref}
\usepackage{paralist}
\usepackage{subcaption}
\usepackage{booktabs}
\usepackage{listings}
\usepackage{caption}

%
% todo notes
%

\usepackage{todonotes}
%\usepackage[disable]{todonotes}	% hide all todo notes
%\presetkeys{todonotes}{inline}{}	% show all defined todos as inline

%
% glossary
%

\usepackage[acronym,numberedsection,automake]{glossaries}
\setacronymstyle{long-short}
\loadglsentries{chapters/used_terms.tex}
\loadglsentries{chapters/glossar.tex}
\makeglossaries
\glsdisablehyper
%
% load additional packages
%

\usepackage{calc}
\usepackage{enumitem}
\usepackage{multirow}
\usepackage{mathtools}
\usepackage{enumitem}
\usepackage{amsmath} 
\usepackage{amssymb}
\usepackage{wasysym}
\usepackage{rotating}
\usepackage{pgfplots}
\usepackage{longtable}
\usepackage{algorithm}
\usepackage{algpseudocode}
\usepackage{float}
\usepackage{pdfpages}
\usepackage{threeparttablex}
\usepackage{longtable,lscape}
\usepackage{tablefootnote}
\usetikzlibrary{patterns}
\usepackage{multicol}
\usepackage{bm}
\usepackage{esvect}
\floatname{algorithm}{Algorithmus}


\pgfplotsset{compat=1.16} 
\begin{document}

%==============================================================================

%
% add title pages
%

\frontmatter
\setcounter{secnumdepth}{3}
\begin{titlepage}
\thispagestyle{empty}
\enlargethispage{2cm}
%\enlargethispage{\baselineskip}

\sffamily
%\centering
\vspace*{-3.2cm}
\hspace*{-0.6cm}
\includegraphics[height=2.14cm]{figures/kit-logo.png}

\begin{addmargin}{4cm}

\vfill

\begin{tabular}{p{12cm}}
	{\bfseries\huge Bachelor Thesis}\\
	\\
	{Tim Engbrocks} \vspace{2em} \\
	\\
	{\linespread{0.85}\selectfont \Huge Development of a Concept for the Monitoring of Decentralized Identity Solutions Based on DevOps Concepts\par}   \vspace{0.5em}\\
	{\LARGE } \vspace{0.5em} \\
	{\LARGE }
\end{tabular}
\vfill
\vfill
\vfill


%\singlespacing

\todo{Add correct month}
\vspace{1em}
\begin{tabular}{ll}
	
        
	MONTH 2023 - MONTH 2023 \\
	\\
	First referee: 				& Prof. Dr. Sebastian Abeck \\
	Second referee:				& Prof. Dr. Bernhard Neumair \\
	Supervising employee:		& Stefan Throner \\
	\\
	\multicolumn{2}{l}{Cooperation \& Management (C\&M, Prof. Abeck)} \\
	\multicolumn{2}{l}{Institute for Telematics, Department of Informatics} \\
	\multicolumn{2}{l}{www.cm.tm.kit.edu} \\
	\\
	\multicolumn{2}{l}{\scriptsize{KIT - The Research University in the Helmholtz Association}}\\
\end{tabular}

\end{addmargin}
\newpage
\thispagestyle{empty}
\end{titlepage}
\chapter*{Statement of Authorship}
%\addcontentsline{toc}{chapter}{Ehrenwörtliche Erklärung}
\thispagestyle{empty}

\vspace*{4cm}




I declare that I completed this thesis on my own and that information which has been directly or indirectly taken from other sources has been noted as such.
Neither this nor a similar work has been presented to an examination committee.

\bigskip
\bigskip
\bigskip

Karlsruhe, \today

\bigskip
\bigskip
\bigskip

\rule{0.3\textwidth}{0.4pt}\\
Tim Engbrocks
\tableofcontents

%
% add content
%

\mainmatter
\chapter{Introduction}
\label{cha:introduction}

\todo{Add chapter introduction}

% The following sections suggest an outline proposal for a first chapter of a bachelor/ master thesis written
% by Cooperation \& Management (C\&M) at Karlsruhe Institute of Technology (KIT).  

\section{Introduction to the Topic Area}
% If the work is based on a concrete project scenario, this project scenario should already be introduced
% in this section at a high level of abstraction.

\todo{Introduce topic area}

\section{Research Questions}
% The first research question should be superior to the following questions ''2 to n'' 
% and should be dealt with in the first content chapter which is Chapter 4.

\todo{Come up with research questions}

% The second research question concerns the concrete software system which serves as a (first part of a)
% demonstrator for the overall research question. This software system will be developed in cooperation with
%  the project team (JuniorStudents) which is co-supervised by the author (SeniorStudent) of this thesis.
% Chapter 5 (Technical Foundation) introduces the existing preliminary work which particularly includes the already
% existing artifacts relevant for the software system to be developed. Chapter 6 (First Solution) describes the
% structured development approach of the software system (in the case of a microservice-based application,
% this is C\&M's microservice engineering approach).

% Research questions 3 to n should then be addressed in the following chapters.
% These research questions should be formulated in detail only AFTER clear answers to the second research question have
% been worked out since these results have a strong influence on the further alignment of the thesis.

\section{Description of the Demonstrator}
% The demonstrator should clarify the research questions introduced in the previous section on an appropriate conceptual level.
% The demonstrator introduces a practical example and shows a solution. In its first draft,
% it corresponds to the software system which is developed in cooperation with the co-supervised project team (JuniorStudents).

\todo{Describe demonstrator}

\section{Thesis Structure}
% \label{sec:gliederung}
% This chapter illustrates the structure of the work in the form of the chapter structure.
% The following chapter structure is recommended:

\todo{Explain thesis structure}

% \subsection*{Chapter 2: Foundations}
% This chapter contains information necessary for a basic understanding of the thesis.
% The information is presented as ''value-free'' as possible.

% \subsection*{Chapter 3: State of the Art}
% The results of the Top Literature are an essential part of this chapter. The structure described in the introduction can be recalled here.

% In contrast to Chapter 2, the contents of this chapter are presented in an argumentative and evaluative form.

% \subsection*{Chapter 4: First Content Chapter}
% Chapters 4 to 4+n are the conceptual contribution chapters of the work in which the achieved results are described.

% \subsubsection*{Chapter 5: ...}

% \subsubsection*{Chapter 6: ...}

% \subsection*{Chapter 5+n: Project Team Work}
% The second last chapter contains the results of the project team work.
% The chapter also describes events (e.g. Coding Day, visits to institutes) in which the person working on this thesis has participated.

% \section{Content-related Overview}
% The overview should describe the most important topic dependencies by means of an illustration.

% \section{Further Information}
% \textit{This section provides various notes on language conventions, formatting, or other technical aspects.
% Therefore, this section must be removed before completing the Bachelor's/Master's thesis.}

% \subsection{Inserting PowerPoint Figures}
% Images are exclusively created with PowerPoint in the image document which should contain all figures in the order of their
% appearance in the thesis document. The title of the slide in the thesis figures file document should correspond to the title of
% the figure in the thesis document. The thesis figures file is saved as .png in the Overleaf folder ``figures''.

% Inserting and referencing the figure is done by a reference: Figure \ref{fig:the_fig_fil_pro}.

% \begin{figure}[h]
% 	\centering
% 	\includegraphics[width=0.8\textwidth]{figures/the_fig_fil_pro.png}
% 	\caption{Thesis Figures File Process}
% 	\label{fig:the_fig_fil_pro}
% \end{figure}

% The steps of the process described in Figure \ref{fig:the_fig_fil_pro} should be carried out by each JuniorStudent
% and SeniorStudent when creating the initial version of the thesis document.



\chapter{Foundations}
\label{cha:foundations}

This chapter provides descriptions of the concepts used in this thesis.
Section \ref{sec:foundation_decentralized_identities} describes decentralized identities
which are the focus of the research project BestRentalPoC.
Section \ref{sec:foundation_observability} describes observability and monitoring which are
the focus of this thesis.
Section \ref{sec:foundation_devops} and \ref{sec:foundation_cicd} describe the concepts
of DevOps and CI/CD which are used in the results provided by this thesis.

\section{Decentralized Identities}
\label{sec:foundation_decentralized_identities}

Decentralized identities can be broken up into two parts: DIDs (Decentralized Identifiers) and VCs (Verifiable Credentials).
DIDs identify a subject while VCs make claims about the subject. A DID in combination with a set of VCs represents
a decentralized identity.
The W3C (World Wide Web Consortium) provides the standards necessary for the implementation and realization
of decentralized identities.
The standard concerning DIDs is the W3C Recommendation Decentralized Identifiers (DIDs) v1.0 \cite{W3C-DID}
and the standard for VCs is the W3C Working Draft Verifiable Credentials Data Model v2.0 \cite{W3C-VC}.

Identifiers are used to uniquely identify something by a single value.
An everyday example of this is an address that uniquely identifies a place.
Most types of identifiers require a centralized registration authority that issues the identifiers.
The task of the registration authority is to keep track of issued identifiers to ensure their uniqueness
as well as their validity. DID are identifiers that do not require
a centralized registration authority which is often accomplished with cryptography.
In the case of the W3C standards, a DID is a URI that resolves to a DID document.
A DID document contains the information necessary to authenticate the subject of a DID.
Additionally, a DID document contains information about its DID controller which is the entity
that can alter the DID document. The DID controller might also be the subject of the DID.
To resolve a DID, DIDs are entered into a verifiable data registry which records the location
of a DID document corresponding to a DID. Any system that can resolve DIDs to their DID documents
is a verifiable data registry but verifiable data registries are most commonly implemented as
distributed ledgers, peer-to-peer networks, decentralized file systems, databases, and other
forms of trusted decentralized data storage.

A credential is a set of claims made by an issuer regarding one or more subjects.
Each claim is an assertion regarding a subject.
For example, a credential might be a driving license. In this case, the issuer is a driving license authority,
the subject is the holder of the driving license and the claim is that the subject is legally allowed to drive.
VCs are a type of credential with additional properties.
They are tamper-evident, meaning that attempts to modify a VC can be detected.
Additionally, their authorship can be cryptographically verified which means that anybody can verify a VC's issuer
to check the validity of the VC.

Decentralized identities provide improved privacy by minimizing the amount of PII (personally identifiable information)
that needs to be shared. An example of this is a digital driving license, implemented as a decentralized identity,
where a person does not have to share their PII but a third party can still validate
that they possess a valid driving license. Due to their cryptographic nature,
decentralized identities also provide tamper-proof identities along with increased data security.

The following is an example of the usage of decentralized identities.
Alice is a citizen who wants to rent a car. Alice has a DID that uniquely identifies her.
She also has a digital wallet associated with her DID that she can use
to store VCs. Bob is a clerk for a driving license authority.
In order for Alice to be able to rent a car online, she needs a method of proving that she has a valid
driver's license. She goes to Bob and asks him to issue her a Verifiable Credential (VC) for her driving
license, which she could use as proof of her possession of a valid driver's license.
Bob verifies that Alice has a valid driver's license and then issues her the VC using her DID.
The VC is stored in Alice's digital wallet which only she has access to.
When she now goes through the process of renting a car online, the rental company
can ask her to provide a VC to prove that she has a valid driver's license.
Alice can then share her VC from her wallet with the rental company, which can then be
verified by the rental company. The rental company now knows that Alice has a valid driving license.
Alice can now proceed with renting a car online from the rental company.

\section{Observability and Monitoring}
\label{sec:foundation_observability}

In 1960, Kalman \cite{Ka60} defined a system to be observable if its exact state at any time can be completely determined
by its outputs. Based on this definition, observability
for modern cloud systems is the practice of capturing outputs from a software system to
infer knowledge about its state. According to Usman et al. \cite{UF+22}, observability consists of capturing
three main types of data from a system: logs, metrics and traces, these are often called the
three pillars of observability.

Logs are streams of textual information emitted by an application. They can contain information about important
events, like an incoming request, or provide details about the occurrence of exceptions.
Traces are a way of tracking the path of requests through a system. A trace contains detailed information
about all the services that were called during the processing of a request and can be thought of
as a stack trace for microservices.
Metrics are numerical data captured from an application from which the state of the application
can be gauged. An example of a metric is the memory usage of an application.
Because of the scope of this work, from now on, only metrics will be considered for observability.
According to Beyer et al. \cite{BJ+16}, monitoring can be used to capture the four golden signals of
a network-based application. These four signals are Latency, Traffic, Errors, and Saturation.
Latency measures the time that it takes the application to service a request starting from its arrival.
Traffic measures the number of incoming requests per second. Errors are the percentage-based rate at which
incoming requests result in an error. Saturation is a measurement of how much capacity of a constrained
resource like memory is being used and how much is still free.

\begin{figure}[tb]
	\centering
	\includegraphics[width=0.7\textwidth]{figures/2.1_observability_and_golden_signals.png}
	\caption{Observability and the Four Golden Signals}
	\label{fig:observability_and_golden_signals}
\end{figure}

Beyer et al. \cite{BJ+16} divide monitoring into two categories: white-box and black-box monitoring.
A monitoring approach belongs to the white-box category when it is only based on information that is
internal to the system being monitored. Contrary, a monitoring approach belongs to the black-box category
when it is only based on the externally visible behavior of the system.
Internal resources describe resources that allow direct access to their internals.
An example of an internal resource is a self-hosted microservice.
External resources describe resources that do not allow direct access to their internals.
An example of an external resource is a Software-as-a-Service (SaaS) product which is used in a context that should be monitored.
External resources can, because of their nature, only be monitored using the Black-box approach.
Internal resources however can be monitored with both the White-box and Black-box approach.

Aceto et al. \cite{AB+12} define eight motivations for monitoring cloud applications:
Capacity and Resource Planning, Capacity and Resource Management, Data Center Management,
SLA Management, Billing, Troubleshooting, Performance Management, and Security Management.
Capacity and Resource Planning concerns the resources needed to run an application.
Capacity and Resource Management is Capacity and Resource Planning during the operations
of an application. Metrics allow operators to know if more resources are needed.
Data Center Management is mainly concerned with the efficient usage of resources.
One key measurement of the efficiency of a data center is its energy efficiency.
SLA Management refers to the monitoring of parameters, which are defined in Service Level Agreements (SLA).
These parameters must be within set bounds for an SLA to be considered fulfilled.
Billing refers to the monitoring of parameters that influence the cost of running an application.
When an application is hosted on a cloud provider's Infrastructure-as-a-Service (IaaS) system,
one of those parameters might be the number of compute instances that an application uses.
While Troubleshooting usually refers to the tracing of requests and failures to provide a dataset for analyzing and fixing issues in an application,
Monitoring can also be used to aid in troubleshooting by recording the number of failed requests and their context.
Performance Management captures the performance of an application with metrics like
latency to gauge if the application performs as expected or needs adjustments.
Security Management tracks security-related issues with metrics to allow developers to understand
the source of security issues to mitigate them.

\section{DevOps}
\label{sec:foundation_devops}

DevOps (Development and Operations) is a philosophy for developing software.
It provides practices and principles which aim to enhance the collaboration
between software development and its operation which in turn should increase
the overall efficiency of the software development process.

DevOps is based on the agile approach to project management.
Agile project management turns linear processes into iterative ones.
An example of a linear process is the waterfall model for software development in which
the different phases of development happen one after another.
In an iterative process, the work is split into multiple packages
which are completed in order. For each process, there is a complete linear work process.
This results in multiple processes which are iteratively completed.
The principles of agile project management can also be applied to software development
to get an iterative development process.
DevOps takes this iterative approach and expands on it through faster cycle times of
building and shipping a new version of an application.
To achieve this, DevOps focuses on increasing the efficiency of cross-functional
collaborations between developers and operators with four core principles \cite{GIT-DEV}.
The first principle is automation. According to DevOps, everything that is needed
to deliver software to a customer should be automated. On the development side, this includes
testing and building. On the operations side, this includes provisioning infrastructure
and deploying software as well as monitoring it. DevOps commonly employs CI/CD to achieve this.
The second principle is collaboration. Commonly, teams, which participate in the
development process of a software project, are split into two categories:
Development teams, which only focus on the development of the software,
and operation teams, which focus on its deployment and operation.
To increase the cooperation between participants, DevOps reorders teams to have both
developers and operators. In this configuration, each team is responsible for the whole process
of its contribution from development to operations.
The third principle is the focus on continuous improvements and minimization of waste
in the form of time and money. DevOps employs monitoring and the measurement of metrics
for this task.
The fourth principle is the focus on the customer's needs. One advantage of an iterative process
compared to a linear one is that changes requested by a customer can be adopted faster.
In a linear process, the whole process would have to be started from the beginning to adopt changes.
An iterative process can adopt changes in its next iteration step which, compared
to the overall length of the complete process, takes less time.
Lwakatare et al. \cite{LK+16} sort these principles into the five dimensions
Collaboration, Automation, Culture, Monitoring, and Measurement where the third DevOps
principle spans the dimensions of Monitoring and Measurement. The dimension of
Culture adds additional focus on the principle of collaboration by stating that
joint responsibility is necessary for the success of the collaboration between
developers and operators. Joint responsibility ensures that all persons involved
have the common interest of maintaining the quality of the software that they are developing.
The combination of these principles and dimensions into DevOps promises faster
development times for software while simultaneously improving its quality thus
resulting in an overall efficiency gain for the development process.

\section{CI/CD}
\label{sec:foundation_cicd}

CI/CD (Continuous Integration/Continuous Deployment) is a part of DevOps
and combines continuous integration with continuous delivery and deployment to
achieve the DevOps principle of automation. Generally speaking,
CI/CD tries to reduce the amount of human involvement needed after a developer has written
source code. This means automating the building and testing of that code which is referred
to as continuous integration as well as the deployment and infrastructure provisioning
for the deployment which is continuous deployment. Sometimes CD also refers to
continuous delivery instead of deployment. In this case, CI/CD stops before the deployment
of the software and ends with end-to-end tests of the system after integrating changes.

CI/CD has eight fundamental elements \cite{GIT-CICD}.
The first two elements are a single source repository and frequent check-ins to the main branch.
This practice reduces the number of duplicated artifacts, makes configuration easier and
works to reduce the number of merge conflicts.
Elements number three and four are automated and self-testing builds.
Automated builds reduce the amount of human involvement necessary while automated testing ensures
the software's quality. Element five supports automated testing with stable testing environments
that are clones of the live production environment.
Element six adds to the second element with frequent iterations which reduces the size of changes
from each iteration making it easier to identify problems and roll them back.
Element seven is maximum visibility. Every developer should be able to access all artifacts from
the development process to make the identification of concerns easier and more likely.
The last element is predictable deployments at any time. This ensures that should a problem arise
in a production environment, it can easily be rolled back with confidence.
Overall this approach promises faster times until the customer receives value, less context switching
from working with multiple different sources and fast recovery in case of problems.

\chapter{State of the Art}
\label{cha:state_of_the_art}

This chapter contains descriptions of the three most relevant papers for this thesis that are related to the topic
of monitoring in the context of microservices and cloud applications. In Section \ref{sec:cm_literature}
the three papers are sorted into the C\&M LITERATURE with a brief description of their content
and why they should be included in the C\&M LITERATURE. Section \ref{sec:uf+22}
contains a more detailed description of the paper that is most relevant to this thesis
and how it was used in this thesis.

\section{Analysis of and Contributions to the C\&M Literature}
\label{sec:cm_literature}

The C\&M LITERATURE is a list of publications that is maintained by the C\&M research group.
The publications contained in this list are relevant to the different areas of research of the group.
The list is split into five topics. Figure \ref{fig:categories_subcategories} shows an overview of these five categories and
sorts the three papers, described in this section, into these categories. The first topic is Engineering which contains publications
by the C\&M research group itself as well as publications that relate to the different phases
of the general software development process. The second topic is Analysis, Architecture, and Testing.
This topic contains publications that are relevant for modeling software and testing it.
The third topic, Authorization and Policies, focuses on access control for microservice-based applications.
The fourth topic, CI/CD and DevOps, contains publications related to the DevOps concept,
cloud-based software, and observability. Lastly, the fifth topic contains a mix of different publications
that were deemed relevant for the C\&M LITERATURE but do not fit into the other topics.

\begin{figure}[tb]
    \centering
    \includegraphics[width=\linewidth]{figures/3.1_literature.png}
    \caption{Categories and Subcategories}
    \label{fig:categories_subcategories}
\end{figure}

\subsection*{A Survey on Observability of Distributed Edge \& Container-Based Microservices \cite{UF+22}}

This paper provides a survey of state-of-the-art monitoring solutions.
The paper surveyed different monitoring solutions from academic papers as well as
commercial options. The monitoring solutions were analyzed regarding their scope,
their structure (multi-tenant and multi-layer), the instrumentation they provide (metrics, logs, and tracing),
and whether they are open-source solutions or not. The survey is summarized in a large table
which is followed by a description of each monitoring solution that was surveyed.
Additionally, the paper defines the data needed by an observability system, its basic functionalities,
and its key characteristics. For the context of C\&M, the paper provides a good overview of
both academic and commercial monitoring systems which can be used to investigate
different monitoring solutions in the future.

Status: CREATE (the publication should become a part of the C\&M LITERATURE  (09.09.2023))

\subsection*{Site Reliability Engineering \cite{BJ+16}}

This book by Beyer et al. \cite{BJ+16} discusses how Google develops reliable software
at some of the largest known scales. The authors of the book work in the Site Reliability Team
at Google. Site Reliability Engineering is the practice of using automated tooling
to increase the reliability of large-scale software systems.
The book is divided into five parts. The first part, introduction, introduces
the basic concepts needed by the rest of the book. In the second part, principles,
the book discusses seven principles on which the work of the Site Reliability Team
at Google is based. Part three, practices, goes into detail on the various tasks
of the Site Reliability Team which range from testing and monitoring to incident management.
Part four, management, explains the management aspects of running a Site Reliability Team.
Part five, conclusions, lists lessons learned from other industries and concludes the book.
For the context of C\&M, the book provides interesting insights into large-scale cloud software
systems which are mostly kept as close company secrets. 

Status: CREATE (the publication should become a part of the C\&M LITERATURE  (09.09.2023))

\subsection*{Cloud monitoring: Definitions, issues and future directions \cite{AB+12}}

This paper by Aceto et al. \cite{AB+12} discusses the motivation for monitoring cloud applications
as well as basic concepts and definitions of cloud monitoring. The paper starts by defining
eight tasks of cloud computing for which monitoring is required. This is followed by describing
cloud architecture as consisting of seven layers. Then eight properties are proposed that
a cloud monitoring system should possess. Finally, the paper reviews open issues and future
directions for the research into cloud monitoring. For the context of C\&M,
this paper provides some fundamental definitions for the area of monitoring cloud applications.

Status: CREATE (the publication should become a part of the C\&M LITERATURE  (09.09.2023))

\section{Usman et al.: A Survey on Observability of Distributed Edge \& Container-Based Microservices}
\label{sec:uf+22}

The paper A Survey on Observability of Distributed Edge \& Container-Based Microservices by Usman et al. \cite{UF+22}
was used by this thesis as an entry point into the topic of monitoring cloud-based software.
The paper discusses some of the fundamentals of monitoring like the three pillars of observability
and the four golden signals. Additionally, this paper defines the basic functionalities and key characteristics of an observability system.
While these definitions were not used as explicit goals for the development of the monitoring system in Chapter \ref{cha:first_solution},
they were used as guidelines during the development of the monitoring system and guided different decisions.

While the paper provides important fundamentals for monitoring and observability in general, the key goal of the paper
was a survey of modern observability solutions both from academia and the industry.
In total, the paper compared 43 different solutions based on their scope, whether they support multi-tenancy or multi-layer observability,
which pillars of observability are captured by the solutions and if they are open source or not.
This list was used as a basis for selecting the different components of the monitoring system developed in Chapter \ref{cha:first_solution}.


\chapter{Overall Concept of the Developed Solutions}

\todo{Add chapter introduction}

% In this first result chapter of the thesis the overall concept which clarifies the relationship of the investigated issues is introduced.
% From this overall concept all further result chapters and the solutions covered by each of these chapters should be derived.

\section{Monitoring BestRental}

\subsection{Monitoring Targets}

The Demonstrator for BestRental includes multiple types of resources.
These can be grouped into internal and external resources.
\begin{enumerate}
    \item Microsoft Application: External
    \item Service: Internal
    \item Cluster: Internal
    \item Azure Tenant: External
    \item Azure Resource Group: External
    \item Azure Directory: External
    \item Service (Principal): External
    \item Trust System: External
    \item Trust Resolver: External
\end{enumerate}

For the collection of metrics from external Microsoft Azure resources, the service Azure Monitor can be used.
This service allows the export and import through a REST API.

\chapter{Technical Foundation}

\todo{Add chapter introduction}

% In this chapter, the technical foundation of the solutions developed in this thesis is worked out. 
% Usually, this is the first chapter written by each SeniorStudent since it describes the current (technical) status 
% the further work of this thesis is based on. 
% By doing so, the continuity of the work carried out by the research group C\&M is guaranteed.
% Additionally, this chapter deals for those JuniorStudents who are co-supervised by the 
% SeniorStudent (the author of this thesis) as one of the most relevant sources for the JuniorStudents' practical work.

% Depending on the concrete topic of the thesis, the technical foundation may include the 
% (i) software and/or system architecture of the software system under investigation, 
% (ii) the artifacts relevant for the thesis, 
% (iii) tools that are applied,
% (iv) any further technical system or solution. 
% These parts of the technical foundation should be described from the viewpoint of the specific topic of this thesis. 
% For example, if an artifact is relevant for the topic, only the topic-related aspects 
% of this artifact (and not just the artifact) should be illustrated in this chapter.

% General description of concepts or solutions are NOT part of this chapter, 
% but should be transferred to other chapters of the thesis (e.g., Chapter 1 or Chapter 2). 
% This also holds for the description of concrete solutions which are NOT to be described in Chapter 5
%  but in the following chapters. The focus of Chapter 5 is to make clear what is missing in the 
% current technical solution in order to motivate the work which will be carried in this thesis 
% (and which will be further described in the following chapters).

% Relevant sources for the content of Chapter 5 are: 
% (i) WASA lecture, 
% (ii) latest Bachelor/Master/PhD theses, 
% (iii) latest publication. 
% The WASA lecture contains the current and, therefore, the ``valid version'' of concepts and solutions. 
% So this content should be trusted if there is a mismatch between WASA lecture and theses or publications.

\section{Monitoring System}
\todo{Finish}
A complete monitoring system requires three main parts: Data sources, data sinks, and the ability
to analyze/visualize the collected data.
The collected data can be split into three different types: Metrics, Logs, and Traces.
This work only provides a monitoring solution for metrics.

Data sources are the origin of the collected metrics and can be split into two different types
based on how they acquire metrics. The first type is data sources that perform manual instrumentation.
This means that the application source code is amended by code that collects metrics and emits them.
Manual instrumentation is useful for application-specific metrics, like for example, how often a user
uses a certain feature in the application.
The second type is data sources that perform automatic instrumentation.
These data sources collect metrics from applications or the environment without changing any source code.
In contrast to manual instrumentation, automatic instrumentation is limited to collecting
common metrics that are provided by an application and its environment like CPU or memory usage.

Data sinks store the metrics that were collected by the data sources.
It is important to note that some data sinks specialize in which type of data they store
to increase efficiency. The last part is the analysis and visualization of the collected metrics.

Optionally, a monitoring system might employ additional components such as data transformers or collectors.
Data transformers can be used to transform data into different formats, enrich it with additional information
or aggregate it.
Data collectors can be used to buffer metrics that were sent by data sources before they are stored in a data sink,
which decreases the load on the data sink.

\section{Grafana}
\todo{Write}

\section{Prometheus}
\todo{Write}

\section{Grafana Mimir}
\todo{Write}

\section{MinIO}
\todo{Write}

\section{Grafana Agent}
\todo{Write}

\chapter{First Solution (Chapter Title is to be Adapted)}

% This is the third result chapter which presents the first solution of the thesis. The solution is integrated into the overall concept introduced in Chapter 4 and it is technically based on the foundation described in Chaper 5.

% A thesis should provide at least two such solutions. 


\chapter{Second Solution (Chapter Title is to be Adapted)}

\todo{Add chapter introduction}

% This is the fourth result chapter which presents the second solution of the thesis. The solution is integrated into the overall concept introduced in Chapter 4 and it is technically based on the foundation described in Chaper 5.


\chapter{Practical Task}

\todo{Add chapter introduction}

% In this chapter the results of the practical tasks carried out in cooperation with a research partner are reported. The chapter should be self-contained meaning that all relevant aspects of the task, such as the foundation and state of the art of the investigated topics should be an integral part of this chapter.


\chapter{Organization of the Project Team}
\label{cha:projektteam-arbeiten}

\chapter{Summary and Outlook}
\label{cha:outlook}

\todo{Add chapter introduction}

%
% add appendix
%

\chapter{Appendix}
\label{cha:appendix}
\chaptermark{Appendix} 

\section{WASA2 Contributions}

\includegraphics[height=\textheight]{pdfs/engbrocks_wasa2_monitoring.pdf}
\includepdf[pagecommand={\thispagestyle{plain}}, pages={2-last},height=\textheight]{pdfs/engbrocks_wasa2_monitoring.pdf}

\section{Snack and Learn}

The Snack and Learn talk by iCC (iC Consult) \cite{ICC-WEB} on the 29th of June 2023
titled Tools \& Templates - Unleashing the Power of Project Management Templates
discussed what project management templates are, their use cases, and their benefits.
The talk was held by Eric Bogard and Thomas Schuster.

The talk started with the introduction of the PM CoE (Project Management Center of Excellence)
which is a department at iCC that provides mentoring and support to project managers.
One of the ways the PM CoE supports project managers is by offering them a variety
of project management templates. Project management templates provide a standardized
basis for the processes and artifacts of project management. Examples include
templates for reports and planning. The templates offered by the PM CoE support
agile, waterfall, and hybrid processes and are available for all phases of
a project's lifecycle. All the templates offered by the PM CoE are related
to the tools with which they should be used. According to a statistic cited, only about 35\% of
projects are successful due to a variety of reasons. The goal of the project management
templates is the reduction of unknown risks by providing proven guidelines
for project management which leverage the combined company's expertise and experience.
This is shown in the nine reasons why, according to the PM CoE, project management templates
are important:
\begin{itemize}
      \item Effective use of project artifacts increases the probability of project success
      \item Reduces risk
      \item Starting point for new and/or less experienced PM's
      \item Consistency across the enterprise
      \item Save time - Why reinvent the wheel
      \item Reinforces best practices
      \item Empower you to do your job better
      \item Improves communication
      \item Tailorable to meet the needs of the project
\end{itemize}
The project management templates are based on successful projects and industry-based
practices. They are offered with iCC or customer branding and are available globally.
Because the PM CoE was only just created, a timeline was presented for when
the first templates would be published throughout the year, starting with the first
template in August and with the goal of publishing five templates by the end of the year.
PM CoE also announced a contest where participants could submit their proposals
for project management templates that will be considered in the future. The prizes
were not yet announced.
The talk ended by presenting the topic of the next Lunch and Learn: Project Health Monitor Tool
on the 28th of September 2023. Afterward, there was also a round of questions from audience members.

\section{Best Practices and Guidelines}

\includegraphics[height=\textheight]{pdfs/ume_gu_implementing_a_microservice_in_golang.pdf}
\includepdf[pagecommand={\thispagestyle{plain}}, pages={2-last},height=\textheight]{pdfs/ume_gu_implementing_a_microservice_in_golang.pdf}


%
% Bibliography
%
\makeatletter
\renewenvironment{thebibliography}[1]
     {\section{\bibname}
      \list{\@biblabel{\@arabic\c@enumiv}}
           {\settowidth\labelwidth{\@biblabel{#1}}
            \leftmargin\labelwidth
            \advance\leftmargin\labelsep
            \@openbib@code
            \usecounter{enumiv}%
            \let\p@enumiv\@empty
            \renewcommand\theenumiv{\@arabic\c@enumiv}}
      \sloppy
      \clubpenalty4000
      \@clubpenalty \clubpenalty
      \widowpenalty4000
      \sfcode`\.\@m}
     {\def\@noitemerr
       {\@latex@warning{Empty `thebibliography' environment}}%
      \endlist}
\makeatother

\bibliography{bt_engbrocks}

\bibliographystyle{cmnat}

%==============================================================================

\end{document}

% \subsection{Inserting UML Diagrams}
% UML diagrams are exclusively created with the tool UMLet. The UMLet files are to be stored in the subfolder ``6.UMLet\_Sources''.
% The UMLet diagram is inserted in the format .png in the thesis figures file. The UMLet diagram illustrated in Figure \ref{fig:exa_sys_arc}
% is taken from the WASA lecture.

% \begin{figure}[ht]
% 	\centering
% 	\includegraphics[width=0.4\textwidth]{figures/exa_sys_arc}
% 	\caption{Example of a SystemPlusSoftware Architecture}
% 	\label{fig:exa_sys_arc}
% \end{figure}

% The naming convention of the UMLet file is <figure\_number>.<tite\_of\_the\_figure> as can be easily understood when taking
% a closer look to the folder ``6.UMLet\_Sources'' in which the example diagram shown in Figure \ref{fig:exa_sys_arc} is stored.

% \subsection{Citations}
% \label{subsec:citations}
% Citations can be included in the thesis in the following way:

% \begin{quote}
% \textit{``A microservice is a cohesive, independent process interacting via messages''}
% \end{quote}
% \begin{quote}
% \textit{``fictive book quote.'', \cite[P.~99]{Be02}}
% \end{quote}

% \subsection{Writing Cucumber Features in LaTeX}
% Features are included in the LaTeX code as a special listing and referenced in the text using the feature \ref{lis:example} command.

% \vspace{0.5cm}
% \begin{lstlisting}[caption = {Example for Development Artifacts in the Thesis}, label = {lis:example}, style = kit-cm, language=] 
% First line of the development artifact
% Second line of the development artifact
% \end{lstlisting}

% \subsection{Inserting Tables}
% Table \ref{tab:example_table} shows an simple example for inserting a table in LaTeX.
% \begin{table}[H] 
%Hint: The [H] is part of the packet float. It stands for ``Here'' which means that the table is placed where it is defined.
% 	\centering
% 	\begin{tabular}{ | l | p{7cm} | }
% 		\hline
% 		\textbf{Heading} & \textbf{Further Heading} \\
% 		\hline
% 		 Entry & Example of an entry \\
% 		\hline
% 		 Entry & Example of a further entry \\
% 	 	\hline
% 	\end{tabular}
% 	\caption{Example of a Table}
% 	\label{tab:example_table}
% \end{table}

% \subsection{Linguistic Conventions in the Context of the C\&M Software Development Process}
% All artifacts created during the C\&M software development process are written in English.
% This also applies to the features created in the analysis phase, since it is assumed that the user of a software system
% developed by C\&M is an English speaker. American English is used throughout the thesis.

% An important quality aspect to be considered in software development is the consistent spelling of the introduced terms during development.
% Two levels must be distinguished here: The level of ("natural", English) language and the level of (formal, development-related) artifacts.
% An example of a term at the language level is "Todo List Management".
% As a C\&M convention that terms are written on the artifact level in the so-called CamelCase notation, in this example "TodoListManagement".