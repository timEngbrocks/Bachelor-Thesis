\chapter{Technical Foundation}
\label{cha:technical_foundation}

\todo{Add chapter introduction}

% In this chapter, the technical foundation of the solutions developed in this thesis is worked out. 
% Usually, this is the first chapter written by each SeniorStudent since it describes the current (technical) status 
% the further work of this thesis is based on. 
% By doing so, the continuity of the work carried out by the research group C\&M is guaranteed.
% Additionally, this chapter deals for those JuniorStudents who are co-supervised by the 
% SeniorStudent (the author of this thesis) as one of the most relevant sources for the JuniorStudents' practical work.

% Depending on the concrete topic of the thesis, the technical foundation may include the 
% (i) software and/or system architecture of the software system under investigation, 
% (ii) the artifacts relevant for the thesis, 
% (iii) tools that are applied,
% (iv) any further technical system or solution. 
% These parts of the technical foundation should be described from the viewpoint of the specific topic of this thesis. 
% For example, if an artifact is relevant for the topic, only the topic-related aspects 
% of this artifact (and not just the artifact) should be illustrated in this chapter.

% General description of concepts or solutions are NOT part of this chapter, 
% but should be transferred to other chapters of the thesis (e.g., Chapter 1 or Chapter 2). 
% This also holds for the description of concrete solutions which are NOT to be described in Chapter 5
%  but in the following chapters. The focus of Chapter 5 is to make clear what is missing in the 
% current technical solution in order to motivate the work which will be carried in this thesis 
% (and which will be further described in the following chapters).

% Relevant sources for the content of Chapter 5 are: 
% (i) WASA lecture, 
% (ii) latest Bachelor/Master/PhD theses, 
% (iii) latest publication. 
% The WASA lecture contains the current and, therefore, the ``valid version'' of concepts and solutions. 
% So this content should be trusted if there is a mismatch between WASA lecture and theses or publications.

\section{Crossplane}

\section{Monitoring System}
\todo{Finish}
A complete monitoring system requires three main parts: Data sources, data sinks, and the ability
to analyze/visualize the collected data.
The collected data can be split into three different types
according to the three pillars of observability: Metrics, Logs, and Traces.
Because this work only focuses on metrics, logs and traces will be ignored for this explanation of monitoring systems.

Data sources are the origin of the collected metrics and can be split into two different types
based on how they acquire metrics. The first type is data sources that perform manual instrumentation.
This means that the application source code is amended by code that collects metrics and emits them.
Manual instrumentation is useful for application-specific metrics, like for example, how often a user
uses a certain feature in the application.
The second type is data sources that perform automatic instrumentation.
These data sources collect metrics from applications or the environment without changing any source code.
In contrast to manual instrumentation, automatic instrumentation is limited to collecting
common metrics that are provided by an application and its environment like CPU or memory usage.

Data sinks store the metrics that were collected by the data sources.
It is important to note that some data sinks specialize in which type of data they store
to increase efficiency. The last part is the analysis and visualization of the collected metrics.

Optionally, a monitoring system might employ additional components such as data transformers or collectors.
Data transformers can be used to transform data into different formats, enrich it with additional information
or aggregate it.
Data collectors can be used to buffer metrics that were sent by data sources before they are stored in a data sink,
which decreases the load on the data sink.

\section{Grafana}
\todo{Grafana}
Grafana is a tool for visualizing and querying time-series data.
In Grafana you can build dashboards that visualize time-series data

\section{Prometheus}
\todo{Prometheus}
Prometheus is an open-source monitoring system.
Prometheus can collect metrics by two different methods.
Firstly, it can pull metrics from services via HTTP which is called scraping.
Additionally, it can receive metrics via a push gateway to which services can send their metrics.
Prometheus stores the collected metrics in a time-series database.
A time-series database is a type of database that is optimized for data entries,
which are associated with time. An example of this type of data is metrics and the time when they were collected.
To analyze the collected metrics, Prometheus uses a query language called PromQL.

\section{Grafana Mimir}
\todo{Grafana Mimir}
Prometheus only stores its data on local disks which is
Grafana Mimir offers long-term storage for Prometheus data.

\section{MinIO}
MinIO is an object storage that provides the same interface as Amazon Web Service's (AWS) S3 service.
Object storage system store data as objects with a unique identifier. An object contains the data stored
data itself and additional metadata. Objects can store arbitrary types of data.

Storage systems can be divided into two categories by the type of data that they store.
There are systems for structured data like relational or other types of databases.
Then there are systems for unstructured data which need to be able to store random binary data.
These systems for unstructured data can be divided into three categories: File storage, Block storage, and Object storage.
File storage stores files in a hierarchical structure of folders and files.
Files are identified by their path in the hierarchy and can be accessed by multiple users at a time.
File storage is commonly used as the default storage system on most operating systems.
Block storage stores files as chunks of data with a unique identifier. This increases access performance
and reduces wasted memory from having to layout files contiguously in memory but it also increases complexity.
Block storage is mostly used for systems that need minimal delay and consistent performance which are also
able to expend the extra compute time needed for reassembling files from their separate chunks.
Object storage combines the approaches of File and Block storage. Files are stored as objects that
contain the file's data and additional metadata. They have a unique identifier for access and are laid out
contiguously in memory. Object storage is mostly used for static data that needs to be read often but not written.
This is because Object storage requires the complete retrieval and rewrite of an object when modifying it.
Object storage reduces the complexity of the storage system compared to File and Block storage and increases
performance for its specific use cases. One example of the use of object storage is the storage of
assets for web pages like images or scripts. These assets don't have an overall structure which means
that an unstructured storage system is best suited for their storage.
Additionally, these assets need to be accessed more often than they need to be written.

One of the most popular object storage systems for cloud environments is Amazon Web Service's (AWS)
S3; S3 means Simple Storage Service. AWS S3 is a paid service that implements the S3 interface for object storage.
Contrary to AWS S3, the S3 interface is an open standard.
MinIO was built as a free alternative to AWS S3 which implements the same interface.
This way, MinIO can be used interchangeably with other services that rely on AWS S3 for storage.
For this work, MinIO is used as part of the data sink for the monitoring solution.