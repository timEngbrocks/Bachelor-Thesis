\chapter{Foundations}
\label{cha:foundations}

\todo{Add chapter introduction}

\section{Decentralized Identities}

Decentralized Identities are best explained with an example of how and where they are used.
Alice is a citizen who wants to rent a car. Alice has a decentralized identifier (DID) which is a digital identity
that uniquely identifies her. She also has a digital wallet associated with her DID that she can use
to store digital credentials. Bob is a clerk for a driving license authority.
In order for Alice to be able to rent a car online, she needs a method of proving that she has a valid
driver's license. She goes to Bob and asks him to issue her a Verifiable Credential (VC) for her driving
license, which she could use as proof of her possessing a valid driver's license.
Bob verifies that Alice has a valid driver's license and then issues her the VC using her DID.
The VC is stored in Alice's digital wallet which only she has access to.
When she now goes through the process of renting a car online, the rental company
can ask her to provide a VC to prove that she has a valid driver's license.
Alice can then share her VC from her wallet with the rental company, which can then be
verified through the Trust System. The rental company now knows that Alice has a valid driving license
because they could verify that the VC was valid and associated with Alice's DID.
Alice can now proceed with renting a car online from the rental company.

Throughout this entire process, Alice was in charge of her personal information while
the rental company was still able to verify her possession of a driving license. This is because
the rental company trusts the driving license authority and the Trust System.
An identity system like this that is based upon trust towards certain entities which uses DID's and VC's is called
a decentralized identity system.

The World Wide Web Consortium (W3C) provides a standard for this type of system in their Decentralized Identifiers (DIDs) v1.0
Standard along with the Verifiable Credentials Data Model v1.1 Standard.

\todo{continue writing}

\section{Monitoring}

\todo{Cleanup and actually write this section}

According to \cite{9837035}, observability consists of three pillars: Metrics, Logs, and Traces.
Metrics.
Logs are streams of textual information emitted by an application. They can contain information about important
events, like an incoming request, or provide details about the occurrence of exceptions.
Traces are a way of tracking the path of requests through a system. A trace contains detailed information
about all the services that were called during the processing of a request and can be thought of
as a stack trace for microservices.

Because of the scope of this work, only Metrics will be used. Logs and Traces will still be considered
in the development of a solution so that they may be added in the future.

Because of the scope of this work, only the pillar of Metrics is considered.

\begin{quote}
\textit{``Metrics are numerical representations of data that Ops teams use to determine the overall behavior of a system, service, or network component over time.'', \cite{9837035}}
\end{quote}

The four golden signals \cite{Beyer2016-xi}
\begin{itemize}
    \item Latency: time to service a request
    \item Traffic: requests per second
    \item Errors: rate of failed requests
    \item Saturation: how saturated are constrained resources like memory or I/O
\end{itemize}

White-box vs. Black-box Monitoring \cite{Beyer2016-xi}
\begin{itemize}
    \item White-box: Monitoring based on internal system metrics.
    \item Black-box: Monitoring based on externally visible behavior.
\end{itemize}

Internal vs. External resources in a monitoring context:
Internal resources describe resources that allow direct access to their internals.
An example of an internal resource is a self-hosted microservice.
External resources describe resources that do not allow direct access to their internals.
An example of an external resource is a Software-as-a-Service (SaaS) product which is used in a context that should be monitored.

External resources can, because of their nature, only be monitored using the Black-box approach.
Internal resources however can be monitored with both the White-box and Black-box approach.

Motivations for monitoring cloud applications \cite{6483656}
\begin{itemize}
    \item Capacity and Resource Planning
    \item Capacity and Resource Management
    \item Data Center Management
    \item SLA Management
    \item Billing
    \item Troubleshooting
    \item Performance Management
    \item Security Management
\end{itemize}

\todo{Capacity and Resource Planning}

\todo{Capacity and Resource Management}

Data Center Management is mainly concerned with the efficient usage of resources.
One key measurement of the efficiency of a data center is its energy efficiency.

SLA Management refers to the monitoring of parameters, which are defined in Service Level Agreements (SLA).
These parameters must be within set bounds for an SLA to be considered fulfilled.

Billing refers to the monitoring of parameters that influence the cost of running an application.
When an application is hosted on a cloud provider's Infrastructure-as-a-Service (IaaS) system,
one of those parameters might be the number of compute instances that an application uses.

While Troubleshooting usually refers to the tracing of requests and failures to provide a dataset for analyzing and fixing issues in an application,
Monitoring can also be used to aid in troubleshooting by recording the number of failed requests and their context.

\todo{Performance Management}

\todo{Security Management}

\section{DevOps}

\todo{Explain DevOps}