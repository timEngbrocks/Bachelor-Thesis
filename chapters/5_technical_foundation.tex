\chapter{Technical Foundation}

% In this chapter, the technical foundation of the solutions developed in this thesis is worked out. Usually, this is the first chapter written by each SeniorStudent since it describes the current (technical) status the further work of this thesis is based on. By doing so, the continuity of the work carried out by the research group C\&M is guaranteed. Additionally, this chapter deals for those JuniorStudents who are co-supervised by the SeniorStudent (the author of this thesis) as one of the most relevant sources for the JuniorStudents' practical work.

% Depending on the concrete topic of the thesis, the technical foundation may include the (i) software and/or system architecture of the software system under investigation, (ii) the artifacts relevant for the thesis, (iii) tools that are applied, (iv) any further technical system or solution. These parts of the technical foundation should be described from the viewpoint of the specific topic of this thesis. For example, if an artifact is relevant for the topic, only the topic-related aspects of this artifact (and not just the artifact) should be illustrated in this chapter.

% General description of concepts or solutions are NOT part of this chapter, but should be transferred to other chapters of the thesis (e.g., Chapter 1 or Chapter 2). This also holds for the description of concrete solutions which are NOT to be described in Chapter 5 but in the following chapters. The focus of Chapter 5 is to make clear what is missing in the current technical solution in order to motivate the work which will be carried in this thesis (and which will be further described in the following chapters).

% Relevant sources for the content of Chapter 5 are: (i) WASA lecture, (ii) latest Bachelor/Master/PhD theses, (iii) latest publication. The WASA lecture contains the current and, therefore, the ``valid version'' of concepts and solutions. So this content should be trusted if there is a mismatch between WASA lecture and theses or publications.
