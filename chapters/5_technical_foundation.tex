\chapter{Technical Foundation}

\todo{Add chapter introduction}

% In this chapter, the technical foundation of the solutions developed in this thesis is worked out. 
% Usually, this is the first chapter written by each SeniorStudent since it describes the current (technical) status 
% the further work of this thesis is based on. 
% By doing so, the continuity of the work carried out by the research group C\&M is guaranteed.
% Additionally, this chapter deals for those JuniorStudents who are co-supervised by the 
% SeniorStudent (the author of this thesis) as one of the most relevant sources for the JuniorStudents' practical work.

% Depending on the concrete topic of the thesis, the technical foundation may include the 
% (i) software and/or system architecture of the software system under investigation, 
% (ii) the artifacts relevant for the thesis, 
% (iii) tools that are applied,
% (iv) any further technical system or solution. 
% These parts of the technical foundation should be described from the viewpoint of the specific topic of this thesis. 
% For example, if an artifact is relevant for the topic, only the topic-related aspects 
% of this artifact (and not just the artifact) should be illustrated in this chapter.

% General description of concepts or solutions are NOT part of this chapter, 
% but should be transferred to other chapters of the thesis (e.g., Chapter 1 or Chapter 2). 
% This also holds for the description of concrete solutions which are NOT to be described in Chapter 5
%  but in the following chapters. The focus of Chapter 5 is to make clear what is missing in the 
% current technical solution in order to motivate the work which will be carried in this thesis 
% (and which will be further described in the following chapters).

% Relevant sources for the content of Chapter 5 are: 
% (i) WASA lecture, 
% (ii) latest Bachelor/Master/PhD theses, 
% (iii) latest publication. 
% The WASA lecture contains the current and, therefore, the ``valid version'' of concepts and solutions. 
% So this content should be trusted if there is a mismatch between WASA lecture and theses or publications.

\section{Monitoring System}
\todo{Finish}
A complete monitoring system requires three main parts: Data sources, data sinks, and the ability
to analyze/visualize the collected data.
The collected data can be split into three different types: Metrics, Logs, and Traces.
This work only provides a monitoring solution for metrics.

Data sources are the origin of the collected metrics and can be split into two different types
based on how they acquire metrics. The first type is data sources that perform manual instrumentation.
This means that the application source code is amended by code that collects metrics and emits them.
Manual instrumentation is useful for application-specific metrics, like for example, how often a user
uses a certain feature in the application.
The second type is data sources that perform automatic instrumentation.
These data sources collect metrics from applications or the environment without changing any source code.
In contrast to manual instrumentation, automatic instrumentation is limited to collecting
common metrics that are provided by an application and its environment like CPU or memory usage.

Data sinks store the metrics that were collected by the data sources.
It is important to note that some data sinks specialize in which type of data they store
to increase efficiency. The last part is the analysis and visualization of the collected metrics.

Optionally, a monitoring system might employ additional components such as data transformers or collectors.
Data transformers can be used to transform data into different formats, enrich it with additional information
or aggregate it.
Data collectors can be used to buffer metrics that were sent by data sources before they are stored in a data sink,
which decreases the load on the data sink.

\section{Grafana}
\todo{Write}

\section{Prometheus}
\todo{Write}

\section{Grafana Mimir}
\todo{Write}

\section{MinIO}
\todo{Write}

\section{Grafana Agent}
\todo{Write}