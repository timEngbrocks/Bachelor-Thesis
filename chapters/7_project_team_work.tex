\chapter{Organization of the Project Team}
\label{cha:projektteam-arbeiten}

% TODO: chapter intro

\section{Overview of the WASA2 Lecture and Practical Course}

The WASA2 (Web Applications and Service-oriented Architectures 2) is a lecture for master students
offered by C\&M each summer semester. The lecture is accompanied by a mandatory practical course.
The goal of the practical course is to deepen the understanding of the concepts discussed in the WASA2
lecture. Each participant writes a practical course thesis throughout their participation in the course
which provides the main basis for their final grade. 

% TODO: Cite people?
The participants of the practical course are organized into the teams P1 Microservice Engineering,
P2 Access Management, and P3 DevOps. The team P1 Microservice Engineering focused on contributing
to the BestRentalPoC by implementing the microservice RentalManagement and ensuring its
quality through testing. This team was supervised by Mr. Schneider and Mr. Spielvogel.
P2 Access Management focused on the integration of authorization policies into the demonstrator
project from the winter semester 2022/2023 which was called CCSApp. They also analyzed
different authorization architectures and were supervised by Mr. Sänger.
P3 DevOps focused on the usage of the DevOps concept to automate the provisioning
of Microsoft Azure cloud resources that are needed by the decentralized identity solution
that BestRentalPoC employs. This team was supervised by Mr. Throner, Mr. Gogel, and Mr. Engbrocks.

% TODO: Figure
The practical course has a supervision hierarchy that assigns both the participating students
as well as the members of the C\&M group a role. The professor holding the lecture holds
the role of Head of C\&M and is ultimately responsible for supervising the course.
The Head of C\&M creates the WASA2 lecture.
Each project team is supervised by one PhDResearcher and one or more SeniorStudents.
The PhDResearcher is a member of the C\&M group who is currently working on their PhD thesis and is supervised by the Head of C\&M.
The PhDResearchers contribute directly to the WASA2 lecture through their PhD thesis and their publications.
SeniorStudents are students who are currently writing their bachelor's or master's thesis
at the C\&M group and are supervised by the PhDResearchers. They contribute to the PhD thesis' and the publications of the PhDResearchers
through work on their thesis'.
The students who are participating in the practical course have the role
of JuniorStudent. They create their practical course thesis which contributes
to the thesis' of the SeniorStudents.

% TODO: Stundenzettel
To ensure that the JuniorStudents met the expectation of how many hours of work
the practical course requires as set forth by the ECTS (The European Credit Transfer and Accumulation System)
points gained by the practical course, the JuniorStudents have to keep an hour log.
The practical course grants 5 ECTS points which each require 30 hours of work, totalling 150 hours of work.
The hour log also ensures that the JuniorStudents do not spend too much time on the practical
course so that they can also focus on their other lectures and courses.

The teams of the practical course meet in a weekly meeting to discuss their practical course
thesis as well as the status of their tasks. Prior to the weekly meetings, the JuniorStudents
provide a current version of their practical course thesis that is ready to be reviewed
by their supervising SeniorStudents. The SeniorStudents review these documents and provide
feedback which is discussed in the following weekly team meeting. Along with the document,
the JuniorStudents send a weekly status update email to their SeniorStudents and PhDResearcher
to signal them that all of their documents are in order and ready to be reviewed.

In addition to their practical course thesis, the JuniorStudents also create
a project team presentation in which they present their team's work to their fellow JuniorStudents.
This project team presentation takes place during the WASA2 lecture.

% TODO: how to cite team work document? not wasa and not gitlab
The practical course is split into three phases familiarization, supervised work,
and unsupervised work. The first phase, familiarization, spans the first two weeks of the practical
course. During this phase, the JuniorStudents focus on familiarizing themselves with the working
processes of the C\&M group which are documented in the C\&M teamwork document \cite{CM-W-TEA}.
They also familiarize themselves with the demonstrator project of the semester by describing
its current state in their practical course thesis. During the second phase, supervised work,
the JuniorStudents work in close collaboration with their SeniorStudents and their PhDResearcher
to create the basis for their team's project. This basis is then used
by the JuniorStudents in the third phase, unsupervised work, to complete their team's project
without any major assistance from their supervisors.

\section{Project Team P3 DevOps}

% TODO: Link entra
The author of this thesis supervised the project team P3 DevOps that participated in the
WASA2 practical course. Team P3 focused on the provisioning of resources for the BestRentalPoC.

% Konzept für die Phasen:

% Phase 1 Einarbeitung

% Anforderungen erfassen für die Umsetzung des BestRentalPoC
% Einarbeiten in die Grundlegenden Technologien und Verfahren zur Bereitstellung
% Erstellen einer Struktur anhand der Phasen
% Notwendige Voraussetzung schaffen für die Umsetzung

% Phase 2 Konzept und Validierung

% Konzept erarbeiten für die Umsetzung der Provisionierung und Verwaltung
% Nachweisen, dass Konzept angewendet werden kann anhand einfachen Demonstrator

% Phase 3 Anwendung der Konzepte auf BestRental

% Konzept zur Provisionierung auf BestRentalPoC Anwenden
% Integration des Monitoring Konzeptes in den BestRentalPoC

The three phases of the practical course were planned as follows.
During the first phase of familiarization, the JuniorStudents would learn about the required
technologies for provisioning BestRentalPoC and create a concept for its provisioning.
The second phase would focus on creating a sample demonstration implementation
of the provisioning concept from the previous phase and validating it.
The final phase would then see the validated provisioning concept be applied to BestRentalPoC.

% TODO: Entra ref
% TODO: Crossplane Beschreibung?
% TODO: results der aufgaben einbinden und grafiken
The JuniorStudents of team P3 DevOps started the practical course by writing a description
of BestRentalPoC in its current state for their practical course thesis. They also started
to read the documentation of Entra (Microsoft Entra Verified Id) which is the decentralized identity
solution used by BestRentalPoC. In the next step, the students then had to provision
a GitLab repository using Crossplane. Crossplane is the tool that was later used to provision
resources in Microsoft Azure. After this, the team was split into two groups.
The first group, called provisioning, would focus on creating a provisioning concept for BestRentalPoC,
while the second group, called monitoring, would assist the author in integrating SPMonitor into BestRentalPoC.
The provisioning group started by reading the documentation of the various cloud services that are involved
in running an Entra System. They were also tasked with creating a concept
for how the microservices of BestRentalPoC should handle credentials that are required
to interact with Entra. The monitoring group meanwhile familiarized themselves with this thesis as well as
the work of project team P1 Microservice Engineering who were responsible for creating the microservice
DM-Car (Domain Microservice Car) for BestRentalPoC. They also adapted the API diagram of DM-Car
to include the necessary components for monitoring.
After this, the second phase of supervised work started. The provisioning group started this phase
by creating initial drafts of the artifacts required to provision the resources
of Entra using Crossplane. The monitoring group started this phase by creating a PoC (Proof of Concept)
implementation of the adapted API diagram of DM-Car using the Prometheus Golang Client. Because team P1
had already implemented the previous version of the API diagram of DM-Car, the monitoring group
only had to implement the additional components for monitoring DM-Car. As a preparation
for the upcoming tasks, the monitoring group also familiarized themselves with Kubernetes, Helm, and ArgoCD.
The development work of the second phase was then interrupted by the initial planning of the project team presentation
for the WASA2 lecture where the students presented their provisioning concept as well as their adaptation
of DM-Car's API diagram to the other project teams. The team then worked on their project team presentation
parallel to their other tasks after the initial planning of the presentation.

% TODO: Finish task description, results and maybe some figures

% Aufgaben Log:

% Aufgaben:
% - Beschreibung BestRentalPoC
% - Vertraut machen mit entra
% ---
% - Provision a GitLab project with Crossplane
% --
% Provisioning: Ian, Saad, Valentin
% - Einarbeitung KOmponenten DID
% - Ansatz für wie apps credentials bekommen um mit entra zu kommunizieren

% Monitoring: Roman, Saad
% - Einarbeitung DMCar von P1
% - Einarbeitung meine BA
% --
% Monitoring:
% - PoC implementierung API diagram mit prometheus go client
% - einarbeitung kube/helm/argocd

% provisioning:
% - Yamls für die ganzen Crossplane Artefakte
% --
% - PT Präsentation Orga
% --
% - Valentin: App Registry als Helm Chart in RentalManagement und DLM
% - Ian: Key Vault als Helm Chart für BestRentalVerifiedInfra und DLAKaVerifiedIDInfra
% - Saad: Image push zu azure image registry, credentials in pipeline und BW cluster
% - Alle: Einarbeitung ArgoCD für die Pipeline
% --
% - Saad + Ian: Pipeline Integration
% - Valentin: App Registry Helm Chart
% - Roman: SPMonitor in AM-DLM einbinden
% --
% - Roman: ArgoCD deploy step: creds als group envs, docs für entwickler, sso, neue chart aus base und app reg chart + docs für verwendung mit fokus auf secrets
% - Valentin: Staging Deployment DLAKaApp und BestRentalApp als komplette umgebung
% --
% - Valentin: Staging Deploy Helm Chart und ArgoCD, App Reg Chart fertig + perm tabelle, docs und tabelle mit app und perm ids
% - Ian: Key Vault Chart + perms als array und docs, chart push step docs
% - Roman: ArgoCD secret mit repo-cred label für repo zugriff, impl in deploy step, docs für verwendung deploy step
% - Roman: cert-manager, lets-encrypt, ingress-nginx -> crossplane + helm chart für BWCluster mit docs
% --
% -> Abgabe