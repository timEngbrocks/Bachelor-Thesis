\chapter{Summary and Outlook}
\label{cha:outlook}

This chapter provides a summary of this thesis in Section \ref{sec:summary}.
The summary recaps the concepts developed in this thesis and the work
done by the author. Section \ref{sec:outlook} provides an outlook
on future research direction and extensions of the work done in this thesis.

\section{Summary}
\label{sec:summary}

This thesis investigated the two research questions posed in Chapter \ref{cha:introduction}.
The first research question asked how monitoring can be integrated into the
UME (Unified Microservice Engineering) approach. This question was answered in Chapter \ref{cha:concept}.
Chapter \ref{cha:concept} developed a monitoring concept for the UME approach that was
based on the DevOps principle of feedback.
The second research question related to the verification of the first research question
by asking how the extended UME approach can be used to monitor the BestRentalPoC.
Chapter \ref{cha:first_solution} answered this question through the development
of the monitoring solution SPMonitor that was used to monitor the DLAKaApp within the BestRentalPoC.
The development of SPMonitor extended the story of the BestRentalPoC and used the extended
UME approach.

Chapter \ref{cha:projektteam-arbeiten} described the author's contributions to the
practical course of WASA2 in the summer semester of 2023. During the practical course,
the project team, which was supervised by the author of this thesis, helped with the development
of the SPMonitor Library. The project team then also integrated the SPMonitor Library
into the microservice AM-DrivingLicenseManagement by the project team.

Chapter \ref{cha:m2go} described the author's contribution to the preparation
of the M2Go (Microservice2Go) practical course for the winter semester of 2023/2024.
The contributions of the author to M2Go consisted of adapting the microservice
DM-Car to utilize a PostgreSQL database as its storage and creating exercises
for the third part, Microservice Engineering, of the practical course.
The author developed exercises and their solutions for the implementation
of the microservice DM-Car.

\section{Outlook}
\label{sec:outlook}

The most important result of this thesis is the extension of the UME (Unified Microservice Engineering)
approach with a monitoring concept. This integrates the DevOps principle of feedback
into the UME approach. Further work is required in validating the developed concept.
This can be done by applying the concept to the development of more applications
and by qualitatively comparing it to other monitoring concepts for DevOps-based
software development processes. Additionally, the assumptions made by the monitoring concept
should be further investigated. The concept relies on the assumptions that a viable monitoring
solution already exists prior to the development of the application
and that the monitoring solution collects metrics from the application by pulling new data
through monitoring endpoints on the application. These assumptions can be investigated by
using the concept with different monitoring solutions and also testing push-based monitoring solutions
as opposed to pull-based solutions.

This thesis developed the monitoring solution SPMonitor
for the DLAKaApp. There are four ways in which SPMonitor
can be extended in the future. Firstly, SPMonitor could be configured
to capture more metrics from DLAKaApp, especially technical metrics.
Currently, SPMonitor only captures the memory usage of the DLAKaApp through the
MemUse metric. Additional technical metrics that might be of interest
to the DLAKaApp in the future are the CPU and storage usage as well as the
four golden signals that were defined in Chapter \ref{cha:foundations}.
Secondly, SPMonitor could be integrated into the BestRentalApp to monitor
the complete BestRentalPoC.
Thirdly, SPMonitor could be extended to a complete observability system
that also captures the missing two pillars of observability logs and traces.
Fourthly, SPMonitor could be used for researching business processes within
the DLAKaApp by integrating features like alerting. Alerting sends out
automated notifications when metrics exceed pre-configured values.
This can be used in conjunction with DevOps to, for example,
automate the horizontal scaling of the DLAKaApp based on metrics captured by SPMonitor.