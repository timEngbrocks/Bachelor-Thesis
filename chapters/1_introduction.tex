\chapter{Introduction}
\label{cha:introduction}

\todo{Add chapter introduction}

% The following sections suggest an outline proposal for a first chapter of a bachelor/ master thesis written
% by Cooperation \& Management (C\&M) at Karlsruhe Institute of Technology (KIT).  

\section{Introduction to the Topic Area}
% If the work is based on a concrete project scenario, this project scenario should already be introduced
% in this section at a high level of abstraction.

\todo{Introduce topic area}
\subsection{Story BestRentalPoC}
\Gls{Alice} is a customer of \Gls{BestRental} and has been renting cars from them.
However, every time she rented a car, she had to remember to bring her physical driving license.
Fortunately, DLAKa, the driving license authority DLA in Karlsruhe (Ka), 
has introduced a digital driving license option. \Gls{Alice} heard the news and decided to go
to the authority to obtain her digital driving license. Bob, who works at DLAKa,
first checks that Alice's physical driving license is valid. Then, \Gls{Bob} fills out a digital form
using the data on Alice's physical driving license. Afterward, \Gls{Alice} scans the QR code on Bob's monitor
using her authenticator app and grants permission to create her digital driving license.
As a result, \Gls{Alice} now possesses a digital driving license in the wallet of her authenticator app.
\Gls{Alice} now can provide digital proof of her valid driving license to anyone who requests it.
Next, \Gls{Alice} rents a car from BestRental. But this time, she uses her digital driving license
to prove that she has a valid one. Once \Gls{Alice} opens the website of BestRental,
she will be prompted to present her digital driving license. Therefore, the website displays a QR Code,
which \Gls{Alice} scans with her authenticator app. Next, \Gls{Alice} confirms the presentation using
her authenticator app and decides what information gets shared. 
Finally, \Gls{Alice} has been verified on BestRental's website and can now rent a car. 
\Gls{Alice} provides the desired period and location for the car rental.
On the website, \Gls{Alice} can then browse and choose from available cars.
Once she has selected a car, \Gls{BestRental} will reserve the car for her.


\subsection{Motivation BestRentalPoC}
The company \Gls{BestRental} wants to develop a monitoring solution for its \Gls{BestRentalPoC} system
to collect metrics that are meant to assist the company in operating the BestRentalPoC.
The system will be operated by QA Engineers, system operators, and a project manager.
Before developing the monitoring solution, the team tasked with developing the solution,
interview the different roles to find out how they will use the solution
and what their tasks and responsibilities are.
QA Engineers are responsible for the reliability of BestRentalPoC.
To ensure the reliability of the system, they want to know when requests to the system fail and
trigger errors, so that they can fix them.
For them to better understand errors, they want to know how often errors occur and from which services,
inside of BestRentalPoC, they originate.
The system operators run the BestRentalPoC. They need to make sure that at any point in time
\Gls{BestRentalPoC} has enough instances of all of its services available to serve all incoming requests.
Because \Gls{BestRental} wants to operate profitably, they can't just create as many instances as they like,
and they need to shut down instances when they are not needed.
To assist them with their task, they want to know how high the resource usage of each instance is
and how many requests are coming into the system. The resources that they want to track for each
instance are the instance's CPU and memory usage.
The project manager's yearly bonus is tied to how profitable the \Gls{BestRentalPoC} operates.
To ensure that he gets his bonus, the project manager wants to track the operating costs of the complete system.
Additionally, he needs to how high the operating costs of each service are, so that he can
allocate development time to increase the efficiency of a service, making it more profitable.
After the interviews, the development notes that all types of users of the monitoring solution
want to be able to do five different things.
Firstly, they want to be able to create a dashboard for a metric that displays its current and historic values.
When viewing a dashboard they also want to be able to run queries on the metric to, for example, get
its value at a specific point in time.
Because the users can't constantly watch all of their dashboards, they want to be able to create
alerts for a metric that will be triggered when the metric exceeds a set value.
When a metric exceeds a value set for an alert, the users want to receive the alert so that they can act on it.
The developers also know that the solution needs to be able to do three main things to accomplish
the requested features.
Firstly, the solution needs to collect metrics from BestRentalPoC,
it then needs to analyze the collected data, and lastly, it needs to conditionally send out alerts
for metrics.

\section{Research Questions}
% The first research question should be superior to the following questions ''2 to n'' 
% and should be dealt with in the first content chapter which is Chapter 4.

\todo{Come up with research questions}

% The second research question concerns the concrete software system which serves as a (first part of a)
% demonstrator for the overall research question. This software system will be developed in cooperation with
%  the project team (JuniorStudents) which is co-supervised by the author (SeniorStudent) of this thesis.
% Chapter 5 (Technical Foundation) introduces the existing preliminary work which particularly includes the already
% existing artifacts relevant for the software system to be developed. Chapter 6 (First Solution) describes the
% structured development approach of the software system (in the case of a microservice-based application,
% this is C\&M's microservice engineering approach).

% Research questions 3 to n should then be addressed in the following chapters.
% These research questions should be formulated in detail only AFTER clear answers to the second research question have
% been worked out since these results have a strong influence on the further alignment of the thesis.

\section{Description of the Demonstrator}
% The demonstrator should clarify the research questions introduced in the previous section on an appropriate conceptual level.
% The demonstrator introduces a practical example and shows a solution. In its first draft,
% it corresponds to the software system which is developed in cooperation with the co-supervised project team (JuniorStudents).

\todo{Describe demonstrator}

\section{Thesis Structure}
% \label{sec:gliederung}
% This chapter illustrates the structure of the work in the form of the chapter structure.
% The following chapter structure is recommended:

\todo{Explain thesis structure}

% \subsection*{Chapter 2: Foundations}
% This chapter contains information necessary for a basic understanding of the thesis.
% The information is presented as ''value-free'' as possible.

% \subsection*{Chapter 3: State of the Art}
% The results of the Top Literature are an essential part of this chapter. The structure described in the introduction can be recalled here.

% In contrast to Chapter 2, the contents of this chapter are presented in an argumentative and evaluative form.

% \subsection*{Chapter 4: First Content Chapter}
% Chapters 4 to 4+n are the conceptual contribution chapters of the work in which the achieved results are described.

% \subsubsection*{Chapter 5: ...}

% \subsubsection*{Chapter 6: ...}

% \subsection*{Chapter 5+n: Project Team Work}
% The second last chapter contains the results of the project team work.
% The chapter also describes events (e.g. Coding Day, visits to institutes) in which the person working on this thesis has participated.

% \section{Content-related Overview}
% The overview should describe the most important topic dependencies by means of an illustration.

% \section{Further Information}
% \textit{This section provides various notes on language conventions, formatting, or other technical aspects.
% Therefore, this section must be removed before completing the Bachelor's/Master's thesis.}

% \subsection{Inserting PowerPoint Figures}
% Images are exclusively created with PowerPoint in the image document which should contain all figures in the order of their
% appearance in the thesis document. The title of the slide in the thesis figures file document should correspond to the title of
% the figure in the thesis document. The thesis figures file is saved as .png in the Overleaf folder ``figures''.

% Inserting and referencing the figure is done by a reference: Figure \ref{fig:the_fig_fil_pro}.

% \begin{figure}[h]
% 	\centering
% 	\includegraphics[width=0.8\textwidth]{figures/the_fig_fil_pro.png}
% 	\caption{Thesis Figures File Process}
% 	\label{fig:the_fig_fil_pro}
% \end{figure}

% The steps of the process described in Figure \ref{fig:the_fig_fil_pro} should be carried out by each JuniorStudent
% and SeniorStudent when creating the initial version of the thesis document.

